\documentclass{article}

\usepackage{amsmath}
\usepackage{ctex}
\usepackage{graphicx}

\title{Notes about ebpf learning}
\author{Mayrain}
\date{\today}

\begin{document}
\maketitle

\section{Introduction}
主要是参照b站上的某个视频,看他们介绍ebpf的功能。\\
ebpf是cbpf的后续。\\
1. risc指令集:11个64位寄存器,以及一些指令集用于操控。我已经在图中标定。\\
2. helpers函数。利用这个函数访问内核数据。(提供内核交互。)\\
3. maps,一个全局变量数组。\\
\\
使用方式和概念\\
4. object pinning,一个概念,利用特定工具和接口将程序或者map加载到内核中。一般适用于持久性使用。\\
5. 尾调用优化,一种编译器优化技术,可以减少函数调用的开销。它的实现方式是,将函数调用的返回地址设置为被调用函数的返回地址。这样就可以避免在被调用函数返回后,再返回到调用函数的返回地址。这样就减少了一次函数调用的开销。\\
\\
6. jit: just in time,即时编译。一种编译方式,将源代码编译成机器码的过程放在运行时进行。这样可以减少编译时间,但是会增加运行时的开销。另外它还可以根据不同环境生成不同机器码,提升了泛用度。\\
7. hardening: 保护bpf程序。\\
\\
> tcpdump:命令行的网络流量分析工具。一般用来抓TCP包。应该和wireshark类似。\\



\end{document}